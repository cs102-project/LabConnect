%% This is file `sample-acmsmall.tex',
%% generated with the docstrip utility.
%%
%% The original source files were:
%%
%% samples.dtx  (with options: `acmsmall')
%% 
%% IMPORTANT NOTICE:
%% 
%% For the copyright see the source file.
%% 
%% Any modified versions of this file must be renamed
%% with new filenames distinct from sample-acmsmall.tex.
%% 
%% For distribution of the original source see the terms
%% for copying and modification in the file samples.dtx.
%% 
%% This generated file may be distributed as long as the
%% original source files, as listed above, are part of the
%% same distribution. (The sources need not necessarily be
%% in the same archive or directory.)
%%
%% The first command in your LaTeX source must be the \documentclass command.
\documentclass[acmsmall,nonacm]{acmart}

\usepackage[utf8]{inputenc}
\usepackage{float}
\usepackage{wrapfig}

\makeatletter
\let\@authors\@empty
\makeatother

%%
%% \BibTeX command to typeset BibTeX logo in the docs
\AtBeginDocument{%
  \providecommand\BibTeX{{%
    \normalfont B\kern-0.5em{\scshape i\kern-0.25em b}\kern-0.8em\TeX}}}

%% Rights management information.  This information is sent to you
%% when you complete the rights form.  These commands have SAMPLE
%% values in them; it is your responsibility as an author to replace
%% the commands and values with those provided to you when you
%% complete the rights form.
\thispagestyle{plain}
\setcopyright{none}
\settopmatter{printacmref=false}
\settopmatter{printfolios=true}
\renewcommand\footnotetextcopyrightpermission[1]{}

%%
%% These commands are for a JOURNAL article.
% \acmJournal{JACM}
% \acmVolume{37}
% \acmNumber{4}
% \acmArticle{111}
% \acmMonth{8}

%%
%% Submission ID.
%% Use this when submitting an article to a sponsored event. You'll
%% receive a unique submission ID from the organizers
%% of the event, and this ID should be used as the parameter to this command.
%%\acmSubmissionID{123-A56-BU3}

%%
%% The majority of ACM publications use numbered citations and
%% references.  The command \citestyle{authoryear} switches to the
%% "author year" style.
%%
%% If you are preparing content for an event
%% sponsored by ACM SIGGRAPH, you must use the "author year" style of
%% citations and references.
%% Uncommenting
%% the next command will enable that style.
%%\citestyle{acmauthoryear}

%%
%% end of the preamble, start of the body of the document source.
\begin{document}

%%
%% The "title" command has an optional parameter,
%% allowing the author to define a "short title" to be used in page headers.
\title{LabConnect}
\subtitle{\textit{Spring 2020/21} \textbf{CS102 Project}\hspace{10cm}Assistant: Haya Shamim Khan Khattak}

%%
%% The "author" command and its associated commands are used to define
%% the authors and their affiliations.
%% Of note is the shared affiliation of the first two authors, and the
%% "authornote" and "authornotemark" commands
%% used to denote shared contribution to the research.
\author{Vedat Eren Arıcan 22002643}
\email{eren.arican@ug.bilkent.edu.tr}

\author{\\Borga Haktan B{\.{I}}len 22002733}
\email{haktan.bilen@ug.bilkent.edu.tr}

\author{\\Berkan Şah{\.{I}}n 22003211}
\email{berkan.sahin@ug.bilkent.edu.tr}

\author{\\Alp Ertan 22003912}
\email{alp.ertan@ug.bilkent.edu.tr}

\author{\\Berk Çakar 22003021}
\email{berk.cakar@ug.bilkent.edu.tr}

%%
%% By default, the full list of authors will be used in the page
%% headers. Often, this list is too long, and will overlap
%% other information printed in the page headers. This command allows
%% the author to define a more concise list
%% of authors' names for this purpose.
\renewcommand{\shortauthors}{LabConnect}

%%
%% The abstract is a short summary of the work to be presented in the
%% article.
% \begin{abstract}
%   A clear and well-documented \LaTeX\ document is presented as an
%   article formatted for publication by ACM in a conference proceedings
%   or journal publication. Based on the ``acmart'' document class, this
%   article presents and explains many of the common variations, as well
%   as many of the formatting elements an author may use in the
%   preparation of the documentation of their work.
% \end{abstract}

% %%
%% The code below is generated by the tool at http://dl.acm.org/ccs.cfm.
%% Please copy and paste the code instead of the example below.
%%
% \begin{CCSXML}
% <ccs2012>
%  <concept>
%   <concept_id>10010520.10010553.10010562</concept_id>
%   <concept_desc>Computer systems organization~Embedded systems</concept_desc>
%   <concept_significance>500</concept_significance>
%  </concept>
%  <concept>
%   <concept_id>10010520.10010575.10010755</concept_id>
%   <concept_desc>Computer systems organization~Redundancy</concept_desc>
%   <concept_significance>300</concept_significance>
%  </concept>
%  <concept>
%   <concept_id>10010520.10010553.10010554</concept_id>
%   <concept_desc>Computer systems organization~Robotics</concept_desc>
%   <concept_significance>100</concept_significance>
%  </concept>
%  <concept>
%   <concept_id>10003033.10003083.10003095</concept_id>
%   <concept_desc>Networks~Network reliability</concept_desc>
%   <concept_significance>100</concept_significance>
%  </concept>
% </ccs2012>
% \end{CCSXML}

  \ccsdesc[500]{Computer systems organization~Embedded systems}

%%
%% Keywords. The author(s) should pick words that accurately describe
%% the work being presented. Separate the keywords with commas.
% \keywords{datasets, neural networks, gaze detection, text tagging}

%%
%% This command processes the author and affiliation and title
%% information and builds the first part of the formatted document.
\maketitle
\thispagestyle{headings}

\section{Introduction}
View this template in "Print Layout" form. To use it, begin by editing the preceding section to include information related 
to your project the report you are writing (for help, press F1 when the text cursor is in a field.) 

\subsection*{Using Styles}

As far as possible, do not change any of the formatting, but rather use the existing styles. For example, place the cursor 
in the text "Using Styles" above. Notice the Style is "Paragraph". Try changing it to "Heading 2", then to "Heading 1". 
Note the numbering of subsequent sections is changed automatically.


\section{Details}
The real work goes here! Replace section titles with something relevant to your report.

\subsection{Sub Sections}
\hfill

\subsection{Using References}

Don’t forget to acknowledge any sources that you make use of. Claiming other people's work and ideas as your own is considered cheating and carries severe penalties. 
Make it very clear what is your work and what help, words, ideas, etc., 
you have taken from elsewhere. Be sure to put quotation marks around any sections of text you copy from elsewhere and add a reference to the original source 
(see \cite{Davenport07} for general information and \cite{Davenport08} and \cite{Davenport09} for examples of in-formation required for book, journal and web-based sources.\\
You can insert references in the text by selecting “Insert|Reference-Footnotes… Endnotes.” 
This template uses sequential numbers for references, the most common format used for technical articles.
The template includes a macro, "CreateReference" which should insert a link in-to the text and a corresponding 
entry into the References section at the end of the document, which you can then edit. You can invoke the macro by pressing 
Control-R (but you may need to "Tools|Unprotect Document" and/or Enable-Macros first!) 
Note: newer versions of Word may now include this reference style—check the help.

\begin{table}[!b]
  \label{tab:freq}
  \begin{tabular}{ccl}
    \toprule
    Criteria&TA / Grader&Instructor\\
    \midrule
     & & \\
     & & \\
     & & \\
     & & \\
     & & \\
     & & \\
     & & \\
  \bottomrule
    Overall &&
\end{tabular}
\end{table}
\subsection{And Outlines?}

Once you understand the basics, you may want to switch to "Outline" view to sort out your ideas before returning to the "Page Layout" view to write the actual content. 
If you learn to use styles, outlines and endnotes properly, then Word sorts out the numbering, formatting, etc. for you. 
Try inserting and deleting some of the references from the text and notice again how the other numbers change automatically. 
Having the machine do the layout and such numbering automatically, enables you to concentrate on what is really important, the content. 
Neat and very professional looking, eh?

\section{Summary {\&} Conclusions}

And finally… don’t forget that Word can help to check your spelling (and grammar!)\\
Maintaining lists of research references that can be reused when writing journal articles 
can be a real pain, especially when citation styles vary so much from journal to journal. 
When you have time, I suggest you look at reference managers (e.g., JabRef for BibTeX, or websites such as CiteSeer), 
as well as other document creation options (e.g., LyX, LaTeX and OpenOffice.)\\
Good Luck.

\bibliographystyle{unsrt}
\bibliography{sample-base}

\end{document}
\endinput
%%
%% End of file `sample-acmsmall.tex'.
