\documentclass[12pt]{article}

\usepackage{titlesec}

\titleformat{\section}
  {\normalfont\Large\bfseries}{\thesection}{1em}{}[{\titlerule[0.4pt]}]
  
\titleformat{\subsection}
  {\normalfont\Large\bfseries}{\thesection}{1em}{}[{\titlerule[0.1pt]}]

\usepackage[utf8]{inputenc}
\usepackage{ulem}
\usepackage[a4paper,margin=1in,footskip=0.25in]{geometry}

\begin{document}

    \section*{Bilkent University CS 102
    Project Proposal}\label{bilkent-university-cs-102-project-proposal}
    
    \subsubsection*{Spring 2020-2021
    Semester}\label{spring-2020-2021-semester}
    
    Our group (from Section 2) contains the members:
    
    \begin{itemize}
        \itemsep1pt\parskip0pt\parsep0pt
        \item
          22002643 - \textbf{Vedat Eren Arıcan}
        \item
          22002733 - \textbf{Borga Haktan Bilen}
        \item
          22003211 - \textbf{Berkan Şahin}
        \item
          22003912 - \textbf{Alp Ertan}
        \item
          22003021 - \textbf{Berk Çakar}
    \end{itemize}
    
    \noindent Our original project proposal, named ``LabConnect'', facilitates
    communication between students, TA's, tutors, and instructors. In the
    background, it is mainly a web application (can be ported to Android
    possibly) that aims to assist CS introductory courses in terms of
    organization and communication. Proposed ideas for features include
    priority queuing for TA zoom rooms. For example, those who have
    completed their labs can be tested using pre-defined (by TA or
    instructor) unit tests and ordered from most complete to least, in order
    to decrease waiting times for students who are done with their labs, and
    to optimize the process in general. TA's can also use the system to see
    previous versions of each student's code in a more practical way,
    similar to real version control managers in spirit. The style guidelines
    put forth by the instructors can be enforced automatically by parsing
    the student's sent code files. Much of the repetitive work that course
    staff need to do can be reduced substantially by automated actions,
    allowing TA's to allocate time for more hands-on help towards students.
    The student experience can be improved further by adding helpful
    features such as personal notes for students and so on. \newline
    
    \noindent Of course, the application and its system will be designed in a modular
    way to allow the course staff to adapt into distinct situations by using
    an admin panel to change settings. Furthermore, features can be expanded
    as the project's development goes on, since there is ample room to be
    creative and help students and instructors alike. In terms of
    technologies to be used, there is also much potential to use a wide
    variety such as web technologies, databases, GUI's, and so on. \newline
    
    \noindent In the event that our original proposal is refused, the following
    options listed in the ``CS 102 Project Descriptions'' document are our
    project preferences in order of priority:
    
    \begin{enumerate}
        \itemsep1pt\parskip0pt\parsep0pt
        \item \textit{Option 4} \rightarrow \textnormal{Application for sharing items.}\)
        \item \textit{Option 8} \rightarrow \textnormal{Coding practice system.}\)
        \item \textit{Option 2} \rightarrow \textnormal{Organizer for travelers.}\)
    \end{enumerate}
    

\end{document}