% Use either your LaTeX editor or latexmk to compile.

\documentclass[a4paper, 12pt]{article}

% Document quality things
\usepackage[utf8]{inputenc}
\usepackage{microtype, xcolor}
\usepackage{url, hyperref}
\hypersetup{colorlinks=true, linkcolor=blue, citecolor=black, urlcolor=blue}

% Image-related packages
\usepackage{graphicx}
\usepackage{float}
\graphicspath{ {./ui-design-mockups/} }
\usepackage[font=small,skip=5pt]{caption}

% Setting margins
\usepackage[a4paper, left=2cm, right=2cm, top=1.75cm, bottom=1.75cm, includefoot]{geometry}

% Table helper packages
\usepackage{multirow, multicol}
\usepackage{makecell}
\usepackage{array}
%\usepackage{tabularx} % Not needed currently, but has a few nice options
\usepackage{wrapfig} % Floating figures/tables

% Prevents spamming tedious newlines everywhere, also disables auto indentation, etc.
\usepackage[skip=0.75\baselineskip plus 2pt]{parskip}

% Self-explanatory
\usepackage{titlesec}
\titleformat{\section}[block]{\normalfont\scshape\Large}{\thesection}{1em}{}
\titleformat{\subsection}{\normalfont\large}{\thesubsection}{1em}{}

% Referencing
\usepackage[backend=bibtex, style=numeric-comp, sorting=none]{biblatex}
\addbibresource{bibliography.bib}

\begin{document}
    
    % Header Table
    \begin{table}[h!]
        \renewcommand{\arraystretch}{3}
        \centering
        \begin{tabular}{ | >{\raggedleft\arraybackslash}m{3cm} l >{\raggedleft\arraybackslash}m{3cm} m{3cm} | }
            \hline
            \Huge CS 102 & \textit{Spring 2020/21} & \multirow{2}{*}{\makecell{Project\\Group}} & \multirow{2}{*}{\textbf{\Huge G2C}} \\
            \makecell[r]{Instructor:\\Assistant:} & \makecell[l]{\textbf{Aynur Dayanık}\\\textbf{Haya Shamim Khan Khattak}} & & \\
            \hline
        \end{tabular}
    \end{table}
    
    % Grading Table
    \begin{table}[h!]
            \renewcommand{\arraystretch}{1.4}
            \centering
            \footnotesize
            \begin{tabular}{ l p{1.5cm} | p{1.5cm} | }
                \hline
                \multicolumn{1}{|c|}{\textbf{Criteria}} & \multicolumn{1}{c|}{\textbf{TA/Grader}} & \multicolumn{1}{c|}{\textbf{Instructor}} \\ \hline
                \multicolumn{1}{|p{10.5cm}|}{Presentation} &  &  \\[10ex] \hline
                \multicolumn{1}{r|}{\textbf{Overall}} &  &  \\
                \cline{2-3}
            \end{tabular}
    \end{table}
    
    % Project Information Header
    {\centering\Huge \bfseries \raisebox{0.5ex}{\texttildelow} LabConnect \raisebox{0.5ex}{\texttildelow} \par}
    
    \begin{table}[h!]
        \renewcommand{\arraystretch}{1.4}
        \centering
        \small
        \begin{tabular}{ r l }
            \textbf{Borga Haktan Bilen} & 22002733 \\
            \textbf{Vedat Eren Arıcan} & 22002643 \\
            \textbf{Berkan Şahin} & 22003211 \\
            \textbf{Berk Çakar} & 22003021 \\
            \textbf{Alp Ertan} & 22003912 \\
        \end{tabular}
    \end{table}
    
    % Document Type Header Table
    \begin{table}[h!]
        \renewcommand{\arraystretch}{1.5}
        \centering
        \begin{tabular}{ |>{\centering\arraybackslash}m{15.15cm}| }
            \hline
            \Large \textbf{UI Design Report} \\
            \small (version 1.0) \\
            \small \textbf{\today} \\
            \hline
        \end{tabular}
    \end{table}
    
    % Document begins here...
    
    \section{Introduction to LabConnect}
    
    LabConnect is a developing project that aims to make education more productive for students,
    and more efficient for teaching staff, among other benefits. The feature list compiled for the
    sake of this goal includes items such as:
    \begin{itemize}
        \item Queueing system for live sessions to optimize wait times and student-TA communication
        \item Dashboard designed with a pragmatist mindset, to lessen confusion as much as viable
        \item Instructor panel where new assignments can be added with great flexibility
        \item Analysis view for students and teaching staff alike, to monitor course progress
        \item Announcements board where the teaching staff can reach out to students with ease
        \item Simple one-to-one messaging capability for the sake of light written communication
        \item Note-taking panel for students to take concise notes regarding individual assignments
        \item Detailed view of submission versions where students and the teaching staff can observe
            automated testing results
    \end{itemize}
    
    Though the above is not an exhaustive list of features, it does nonetheless capture the gist of the features
    this project proposes in order to undertake its goal of optimizing the assignment portion of
    a computer science course. For a more extensive summary of this project, refer to the requirements report published earlier.
    
    \section{Disclaimer Regarding the UI Design Report}
    
    The document herein contains details and illustrations from 13 application views in total, but certain disclaimers have to be made
    regarding the accuracy of these illustrations.
    LabConnect is planned to be a web application, built with established modern web design paradigms in mind. However, web pages,
    particularly those that strive to be designed responsively for the sake of usability on a distinct range of devices, are
    not easy to make \emph{static} prototype designs of. Along with this factor, another aspect affecting the UI design process is the fact
    that as LabConnect is an application with a large volume of interaction between people, which may take place at severely
    differing times, an unavoidable need to display certain elements only in very specific instances appears. In other words, the project
    at hand is of such nature that it cannot be \emph{accurately} put on display before an actual development of the interface, via
    the use of dynamic web technologies such as CSS and JavaScript, is in process. 

    As a side note, the development of the interface also depends directly on the implementation of the feature set,
    as the need for elements on the page will originate from the structures designed on the server application side of the project, which
    are highly liable to change as the back-end code undergoes development. An example of this phenomenon is the analysis view presented to the users,
    which is dependent highly on core features being implemented first, because only then can the data to be put on display be ascertained,
    and the interface thereof finalized.
    
    The UI design of LabConnect was completed with the above considerations in mind, which is to mean that the design was developed
    for the sake of having a guide to refer to when the necessity arises, rather than being developed for the impractical sake of being an accurate
    finalized version of the interface. We believe that this approach will prove to be more advantageous in the long term.
    
    \pagebreak
    
    \section{Map of the Application's Views}
    
    
    
    
    % TODO Add a "sitemap" of pages here, as seen on the document on a.dayanik's website.
    
    
    
    
    \pagebreak
    
    \section{User Interface Designs of Application Views}
    
    \subsection{User-agnostic Views}
    
    This subsection illustrates the pages of LabConnect that are intended to remain mostly unchanged regardless of the user's account level in the system
    (i.e., student, TA, instructor). Having certain user-agnostic views may help to make the interface easier to maintain, similar to how
    reusing existing code is often advised.
    
    \subsubsection{Login}
    
    \begin{figure}[H]
        \centering
        \includegraphics[width=\textwidth]{login}
        \caption{Complete view of the login UI}
        \label{fig:login_full}
    \end{figure}
    
    The view seen above in Figure \ref{fig:login_full} is the page every guest user will see upon visiting the website.
    On the left, a concise preview as to what the project is about, and is capable of, is provided.
    The features are also listed in order to give the user the ability to understand the inner workings of the
    system better, in case it may offer them a better experience utilizing the system later on.
    On the right, a similarly concise login panel is provided, with all of the common and basic features
    such as password recovery and a `Remember me" option. 
    
    It is noteworthy that the guest user is not given a choice to register a new account.
    The reason is that, for the time being, the user database is planned to be modified by the administrators of the system directly.
    The users are intended to login once they have been shared the credentials created for them.
    
    \pagebreak
    
    \subsubsection{Dashboard}
    % Beware:
    % This "Dashboard" section is pretty troublesome with its layout.
    % It seems as if changing a word in here might have the potential of breaking things.
    % If you don't absolutely have to, refrain from modifying anything within
    % this section.
    
    \begin{figure}[H]
        \centering
        \includegraphics[width=\textwidth]{main_dashboard}
        \caption{Complete view of the dashboard UI}
        \label{fig:dashboard_full}
    \end{figure}
    
    The view above in Figure \ref{fig:dashboard_full} is the page every logged in user will see as their `homepage".
    The goal is to provide the user with a simple and tidy overview of things requiring their attention.
    The main two panels, unique to the page, are located in the center column: A list of all assignments and a small
    status panel at the top of the screen. The assignment list panel is the main navigation method into each individual
    assignment. Certain details of each assignment are provided on this page, however, the user needs to click on an
    assignment in order to visit a page with more details and features. To make it clear that the assignment items are
    to be clicked on, an entrance symbol is provided on the rightmost side of the item.
    The rest of the elements visible on this page are recurrent throughout most, if not all, of the other pages.
    As such, their significance and properties will be elaborated on in the following paragraph.
    
    \begin{wrapfigure}[15]{r}{0.34\textwidth}
        \centering
        \vspace{-15pt}
        \includegraphics[width=0.34\textwidth]{navbar_short}
        \caption{Navigation bar, shortened for illustration purposes}
        \label{fig:navbar_short}
    \end{wrapfigure}
    
    The navigation bar (Figure \ref{fig:navbar_short}) is the tool to be used to navigate
    anywhere on the application, as well as to show the user their current location. The bar shows brief information regarding
    their assigned TA and instructor, continued by a list of navigation options. Located at the very bottom of the list is two
    options for logging out and reaching the settings, the latter of which is not defined clearly as of yet. As such, the settings
    page is planned to be used should the need arise during the development process. The part on the bar highlighted in red is 
    a reserved space to display the queue list of a live session in cases where the user is traversing pages other than the assignment page
    where the main queue list is located. By displaying a persistent queue list this way, the user experience is made more flexible as
    the user does not necessarily have to constrain themselves to staying on the assignment page. Also, though the space in the figure
    is quite short, this is only for the purpose of illustration, and the true version of the persistent queue box is much more spacious.
    
    \begin{wrapfigure}{l}{0.46\textwidth}
        \centering
        \vspace{-15pt}
        \includegraphics[width=0.46\textwidth]{messages_panel}
        \caption{Messages mini-panel}
        \label{fig:messages_panel}
    \end{wrapfigure}
    
    The messages panel (Figure \ref{fig:messages_panel}) serves the purpose of showing the user a brief outline of their unread messages
    without having to visit the full messages page. The user can click on any of the messages shown, or the right-pointing arrows on the top right,
    in order to visit their full messages page, where they can read and respond to messages.
    The calendar panel (located under the messages panel, see Figure \ref{fig:dashboard_full}) was not added in detailed manner
    to the design prototypes for the sake of simplicity.
    However, very plainly, its purpose is to show the user their assignment due dates visually by marking the days with due dates. 
    It does not have any advanced capabilities at all, and is not exactly intended to be interacted with.
    
    \pagebreak
    % After this point, you don't need to be worried about breaking the "Dashboard" section.
    
    \subsubsection{Messages}
    
    \begin{figure}[H]
        \centering
        \includegraphics[width=\textwidth]{messages}
        \caption{Complete view of the messages UI}
        \label{fig:messages_full}
    \end{figure}
    
    
    
    
    % TODO Write some details for the messages page.
    
    
    
    
    \pagebreak
    
    
    \subsection{Student-specific Views}
    
    This subsection illustrates the pages involved in the user experience of a student account.
    
    \subsubsection{Assignment Details}
    
    \begin{figure}[H]
        \centering
        \includegraphics[width=\textwidth]{student_assignment_details}
        \caption{Complete view of the assignment details UI from a student's perspective}
        \label{fig:student_assignment_details_full}
    \end{figure}

    
    
    
    % TODO Write some details for the assignment details page (student perspective)
    
    
    
    
    
    \pagebreak
    
    \subsubsection{Assignment Submission}
    
    \begin{figure}[H]
        \centering
        \includegraphics[width=\textwidth]{student_assignment_submission}
        \caption{Complete view of the assignment submission UI from a student's perspective}
        \label{fig:student_assignment_submission_full}
    \end{figure}

    
    
    
    
    
    % TODO Write some details for the assignment submission "popup" (student perspective)
    
    
    
    
    
    
    \pagebreak
    
    \subsubsection{Analysis}
    
    \begin{figure}[H]
        \centering
        \includegraphics[width=\textwidth]{student_analysis}
        \caption{Complete view of the analysis UI from a student's perspective}
        \label{fig:student_analysis_full}
    \end{figure}
    
    
    
    
    
    % TODO Write some details for the analysis page (student perspective)
    
    
    
    
    
    
    \pagebreak
    
    \subsubsection{Announcements}
    
    \begin{figure}[H]
        \centering
        \includegraphics[width=\textwidth]{student_announcements}
        \caption{Complete view of the announcements UI from a student's perspective}
        \label{fig:student_announcements_full}
    \end{figure}
    
    
    
    
    
    % TODO Write some details for the announcements page (student perspective)
    
    
    
    
    
    
    \pagebreak
    
    \subsubsection{Notes}
    
    \begin{figure}[H]
        \centering
        \includegraphics[width=\textwidth]{student_notes}
        \caption{Complete view of the notes UI from a student's perspective}
        \label{fig:student_notes_full}
    \end{figure}
    
    
    
    
    
    
    % TODO Write some details for the "my notes" page (student perspective)
    
    
    
    
    
    
    \pagebreak
    
    
    \subsection{TA-specific Views}
    
    This subsection illustrates the pages involved in the user experience of a TA account.
    
    \subsubsection{Assignment Submission}
    
    \begin{figure}[H]
        \centering
        \includegraphics[width=\textwidth]{ta_assignment_submission}
        \caption{Complete view of the assignment submission UI from a TA's perspective}
        \label{fig:student_assignment_submission_full}
    \end{figure}
    
    
    
    
    
    
    
    % TODO Write some details for the assignment submission "popup" (TA perspective)
    
    
    
    
    
    
    
    \pagebreak
    
    \subsubsection{Ongoing Live Session}
     
    \begin{figure}[H]
        \centering
        \includegraphics[width=\textwidth]{ta_live_session}
        \caption{Complete view of the live session UI from a TA's perspective}
        \label{fig:ta_live_session_full}
    \end{figure}
    

    
    
    
    % TODO Write some details for the live session page (TA perspective)
    
    
    
    
    
    
    \pagebreak
    
    
    \subsection{Instructor-specific Views}
    
    This subsection illustrates the pages involved in the user experience of an instructor account.
    
    \subsubsection{Announcements}
     
    \begin{figure}[H]
        \centering
        \includegraphics[width=\textwidth]{instructor_announcements}
        \caption{Complete view of the announcements UI from an instructor's perspective}
        \label{fig:instructor_announcements_full}
    \end{figure}
    

    
    
    
    
    
    % TODO Write some details for the announcements page (instructor perspective)
    
    
    
    
    
    \pagebreak
    
    \subsubsection{Instructor Panel}
    
    \begin{figure}[H]
        \centering
        \includegraphics[width=\textwidth]{instructor_admin_panel}
        \caption{Complete view of the instructor panel UI from an instructor's perspective}
        \label{fig:instructor_admin_panel_full}
    \end{figure}
    
    
    
    
    
    
    % TODO Write some details for the instructor panel (instructor perspective)
    
    
    
    
    
    
    
    \pagebreak
    
    \subsubsection{Analysis}
    
    \begin{figure}[H]
        \centering
        \includegraphics[width=\textwidth]{instructor_analysis}
        \caption{Complete view of the analysis UI from an instructor's perspective}
        \label{fig:instructor_analysis_full}
    \end{figure}
    
    
    
    
    
    
    % TODO Write some details for the analysis page (instructor perspective)
    
    
    
    
    
    
    \pagebreak
    
    \section{Final Remarks}
    
    The user interface of LabConnect was designed while being conscious of the experiences we have been undergoing for the past two semesters of CS courses.
    The same care that we had put into compiling a list of features that we thought would alleviate many of the issues we had observed,
    was put into designing an interface such that users would not be facing the interface as an obstacle at any point during their usage.
    Striving to remain as simple and to-the-point as possible, as the UI design matures throughout the development timeline, the plan is
    to continue to have a focus on being UX-oriented. The design we have formulated is by no means unique, as countless web applications adopt 
    quite similar interfaces. However, rather than being seen as detrimental to the creativity of this design, we consider this wide usage 
    to be a testimony of the design being a viable option for user satisfaction. Additionally, many users may be content with the advantage of
    being familiar with the interface from the very start. 
    
    Also, for the sake of coverage, another point to address is our decision of basing our design on a dark color scheme. Though we are concerned that the psychological association of
    lighter colors with professional-looking reputable websites may surprise some users upon their initial visit, we also firmly recognize that the programmers
    of our day have a strong preference towards interfaces with colors that do not stress the eye. Considering the fact that our project caters quite specifically to
    a user base consisting of programmers, we think that picking a lighter color scheme would have been frustrating to the overwhelming proportion of users who will
    be using this website among their otherwise dark-themed workspace. 
    Alas, we have determined it most sensible to put our efforts into developing a dark-mode interface, though we may choose to add the option of switching
    to a light-mode theme in the later stages of the project's development.
    
    
    
    
    % Not sure if a bibliography will be necessary for this report.
    % Remove later if there's no need.
    \pagebreak
    \printbibliography
    
    
\end{document}

