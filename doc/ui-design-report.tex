% Use either your LaTeX editor or latexmk to compile.

\documentclass[a4paper, 12pt]{article}

% Document quality things
\usepackage[utf8]{inputenc}
\usepackage{microtype, xcolor}
\usepackage{url, hyperref}
\hypersetup{colorlinks=true, linkcolor=blue, citecolor=black, urlcolor=blue}

% Image-related packages
\usepackage{graphicx}
\graphicspath{ {./ui-design-mockups/} }

% Setting margins
\usepackage[a4paper, left=2cm, right=2cm, top=1.75cm, bottom=1.75cm, includefoot]{geometry}

% Table helper packages
\usepackage{multirow, multicol}
\usepackage{makecell}
\usepackage{array}
%\usepackage{tabularx} % Not needed currently, but has a few nice options
\usepackage{wrapfig} % Floating figures/tables

% Prevents spamming tedious newlines everywhere, also disables auto indentation, etc.
\usepackage[skip=0.75\baselineskip plus 2pt]{parskip}

% Self-explanatory
\usepackage{titlesec}
\titleformat{\section}[block]{\normalfont\scshape\Large}{\thesection}{1em}{}
\titleformat{\subsection}{\normalfont\large}{\thesubsection}{1em}{}

% Referencing
\usepackage[backend=bibtex, style=numeric-comp, sorting=none]{biblatex}
\addbibresource{bibliography.bib}

\begin{document}
    
    % Header Table
    \begin{table}[h!]
        \renewcommand{\arraystretch}{3}
        \centering
        \begin{tabular}{ | >{\raggedleft\arraybackslash}m{3cm} l >{\raggedleft\arraybackslash}m{3cm} m{3cm} | }
            \hline
            \Huge CS 102 & \textit{Spring 2020/21} & \multirow{2}{*}{\makecell{Project\\Group}} & \multirow{2}{*}{\textbf{\Huge G2C}} \\
            \makecell[r]{Instructor:\\Assistant:} & \makecell[l]{\textbf{Aynur Dayanık}\\\textbf{Haya Shamim Khan Khattak}} & & \\
            \hline
        \end{tabular}
    \end{table}
    
    % Grading Table
    \begin{table}[h!]
            \renewcommand{\arraystretch}{1.4}
            \centering
            \footnotesize
            \begin{tabular}{ l p{1.5cm} | p{1.5cm} | }
                \hline
                \multicolumn{1}{|c|}{\textbf{Criteria}} & \multicolumn{1}{c|}{\textbf{TA/Grader}} & \multicolumn{1}{c|}{\textbf{Instructor}} \\ \hline
                \multicolumn{1}{|p{10.5cm}|}{Presentation} &  &  \\[10ex] \hline
                \multicolumn{1}{r|}{\textbf{Overall}} &  &  \\
                \cline{2-3}
            \end{tabular}
    \end{table}
    
    % Project Information Header
    {\centering\Huge \bfseries \raisebox{0.5ex}{\texttildelow} LabConnect \raisebox{0.5ex}{\texttildelow} \par}
    
    \begin{table}[h!]
        \renewcommand{\arraystretch}{1.4}
        \centering
        \small
        \begin{tabular}{ r l }
            \textbf{Borga Haktan Bilen} & 22002733 \\
            \textbf{Vedat Eren Arıcan} & 22002643 \\
            \textbf{Berkan Şahin} & 22003211 \\
            \textbf{Berk Çakar} & 22003021 \\
            \textbf{Alp Ertan} & 22003912 \\
        \end{tabular}
    \end{table}
    
    % Document Type Header Table
    \begin{table}[h!]
        \renewcommand{\arraystretch}{1.5}
        \centering
        \begin{tabular}{ |>{\centering\arraybackslash}m{15.15cm}| }
            \hline
            \Large \textbf{UI Design Report} \\
            \small (version 0.1) \\
            \small \textbf{\today} \\
            \hline
        \end{tabular}
    \end{table}
    
    % Document begins here...
    
    \section{Introduction to LabConnect}
    
    LabConnect is a developing project that aims to make education more productive for students,
    and more efficient for teaching staff, among other benefits. The feature list compiled for the
    sake of this goal includes items such as:
    \begin{itemize}
        \item Queueing system for live sessions to optimize wait times and student-TA communication
        \item Dashboard designed with a pragmatist mindset, to lessen confusion as much as viable
        \item Instructor panel where new assignments can be added with great flexibility
        \item Analysis view for students and teaching staff alike, to monitor course progress
        \item Announcements board where the teaching staff can reach out to students with ease
        \item Simple one-to-one messaging capability for the sake of light written communication
        \item Note-taking panel for students to take concise notes regarding individual assignments
        \item Detailed view of submission versions where students and the teaching staff can observe
            automated testing results
    \end{itemize}
    
    Though the above is not an exhaustive list of features, it does nonetheless capture the gist of the features
    this project proposes in order to undertake its goal of optimizing the assignment portion of
    a computer science course. For a more extensive summary of this project, refer to the requirements report published earlier.
    
    \section{Disclaimer Regarding the UI Design Report}
    
    The document herein contains details and illustrations from 12 application views in total, but certain disclaimers have to be made
    regarding the accuracy of these illustrations.
    LabConnect is planned to be a web application, built with established modern web design paradigms in mind. However, web pages,
    particularly those that strive to be designed responsively for the sake of usability on a distinct range of devices, are
    not easy to make \emph{static} prototype designs of. Along with this factor, another aspect affecting the UI design process is the fact
    that as LabConnect is an application with a large volume of interaction between people, which may take place at severely
    differing times, an unavoidable need to display certain elements only in very specific instances appears. In other words, the project
    at hand is of such nature that it cannot be \emph{accurately} put on display before an actual development of the interface, via
    the use of dynamic web technologies such as CSS and JavaScript, is in process. 

    As a side note, the development of the interface also depends directly on the implementation of the feature set,
    as the need for elements on the page will originate from the structures designed on the server application side of the project, which
    are highly liable to change as the back-end code undergoes development. An example of this phenomenon is the analysis view presented to the users,
    which is dependent highly on core features being implemented first, because only then can the data to be put on display be ascertained,
    and the interface thereof finalized.
    
    The UI design of LabConnect was completed with the above considerations in mind, which is to mean that the design was developed
    for the sake of having a guide to refer to when the necessity arises, rather than being developed for the impractical sake of being an accurate
    finalized version of the interface. We believe that this approach will prove to be more advantageous in the long term.
    
    \section{Map of the Application's Views}
    
    
    % TODO
    
    
    \section{User Interface Designs of Application Views}
    
    \subsection{User-agnostic Views}
    
    This subsection illustrates the pages of LabConnect that are intended to remain mostly unchanged regardless of the user's account level in the system
    (i.e., student, TA, instructor).
    
    \subsubsection{Dashboard}
    
    % TODO
    
    \subsubsection{Messages}
    
    % TODO
    
    
    \subsection{Student-specific Views}
    
    This subsection illustrates the pages involved in the user experience of a student account.
    
    \subsubsection{Assignment Details}
    
    % TODO
    
    \subsubsection{Assignment Submission}
    
    % TODO
    
    \subsubsection{Analysis}
    
    % TODO
    
    \subsubsection{Announcements}
    
    % TODO
    
    \subsubsection{Notes}
    
    % TODO
    
    
    \subsection{TA-specific Views}
    
    This subsection illustrates the pages involved in the user experience of a TA account.
    
    \subsubsection{Assignment Submission}
    
    % TODO
    
    \subsubsection{Ongoing Live Session}
    
    % TODO
    
    
    \subsection{Instructor-specific Views}
    
    This subsection illustrates the pages involved in the user experience of an instructor account.
    
    \subsubsection{Announcements}
    
    % TODO
    
    \subsubsection{Instructor Panel}
    
    % TODO
    
    \subsubsection{Analysis}
    
    % TODO
    
    
    \section{Final Remarks}
    
    The user interface of LabConnect was designed while being conscious of the experiences we have been undergoing for the past two semesters of CS courses.
    The same care that we had put into compiling a list of features that we thought would alleviate many of the issues we had observed,
    was put into designing an interface such that users would not be facing the interface as an obstacle at any point during their usage.
    Striving to remain as simple and to-the-point as possible, as the UI design matures throughout the development timeline, the plan is
    to continue to have a focus on being UX-oriented. The design we have formulated is by no means unique, as countless web applications adopt 
    quite similar interfaces. However, rather than being seen as detrimental to the creativity of this design, we consider this wide usage 
    to be a testimony of the design being a viable option for user satisfaction. Additionally, many users may be content with the advantage of
    being familiar with the interface from the very start. 
    
    Also, for the sake of coverage, another point to address is our decision of basing our design on a dark color scheme. Though we are concerned that the psychological association of
    lighter colors with professional-looking reputable websites may surprise some users upon their initial visit, we also firmly recognize that the programmers
    of our day have a strong preference towards interfaces with colors that do not stress the eye. Considering the fact that our project caters quite specifically to
    a user base consisting of programmers, we think that picking a lighter color scheme would have been frustrating to the overwhelming proportion of users who will
    be using this website among their otherwise dark-themed workspace. 
    Alas, we have determined it most sensible to put our efforts into developing a dark-mode interface, though we may choose to add the option of switching
    to a light-mode theme in the later stages of the project's development.
    
    
    
    
    % Not sure if a bibliography will be necessary for this report.
    % Remove later if there's no need.
    \pagebreak
    \printbibliography
    
    
\end{document}

