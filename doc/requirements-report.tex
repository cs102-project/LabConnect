% The order to compile is:
% pdflatex -> bibtex -> pdflatex
% Doing this process with your current working directory
% pointing to where all the tex/bib/etc. files are may help.

\documentclass[a4paper, 12pt]{article}

% Document quality things
\usepackage[utf8]{inputenc}
\usepackage{microtype, xcolor}
\usepackage{url, hyperref}
\hypersetup{colorlinks=true, linkcolor=blue, citecolor=red, urlcolor=blue}

% Setting margins
\usepackage[a4paper, left=2cm, right=2cm, top=1.75cm, bottom=1.75cm, includefoot]{geometry}

% Table helper packages
\usepackage{multirow, multicol}
\usepackage{makecell}
\usepackage{array}
%\usepackage{tabularx} % Not needed currently, but has a few nice options
%\usepackage{wrapfig} % Floating figures/tables

% Prevents spamming tedious newlines everywhere, also disables auto indentation, etc.
\usepackage[skip=0.75\baselineskip plus 2pt]{parskip}

% Self-explanatory
\usepackage{titlesec}
\titleformat{\section}[block]{\normalfont\scshape\Large}{\thesection}{1em}{}
\titleformat{\subsection}{\normalfont\large}{\thesubsection}{1em}{}

% Referencing
\usepackage[backend=bibtex, style=numeric-comp, sorting=none]{biblatex}
\addbibresource{requirements-report.bib}

\begin{document}
    
    % Header Table
    \begin{table}[h!]
        \renewcommand{\arraystretch}{3}
        \centering
        \begin{tabular}{ | >{\raggedleft\arraybackslash}m{3cm} l >{\raggedleft\arraybackslash}m{3cm} m{3cm} | }
            \hline
            \Huge CS 102 & \textit{Spring 2020/21} & \multirow{2}{*}{\makecell{Project\\Group}} & \multirow{2}{*}{\textbf{\Huge G2C}} \\
            \makecell[r]{Instructor:\\Assistant:} & \makecell[l]{\textbf{Aynur Dayanık}\\\textbf{Haya Shamim Khan Khattak}} & & \\
            \hline
        \end{tabular}
    \end{table}
    
    % Grading Table
    \begin{table}[h!]
            \renewcommand{\arraystretch}{1.4}
            \centering
            \footnotesize
            \begin{tabular}{ l p{1.5cm} | p{1.5cm} | }
                \hline
                \multicolumn{1}{|c|}{\textbf{Criteria}} & \multicolumn{1}{c|}{\textbf{TA/Grader}} & \multicolumn{1}{c|}{\textbf{Instructor}} \\ \hline
                \multicolumn{1}{|p{10.5cm}|}{Presentation} &  &  \\[10ex] \hline
                \multicolumn{1}{r|}{\textbf{Overall}} &  &  \\
                \cline{2-3}
            \end{tabular}
    \end{table}
    
    % Project Information Header
    {\centering\Huge \bfseries \raisebox{0.5ex}{\texttildelow} LabConnect \raisebox{0.5ex}{\texttildelow} \par}
    
    {\centering\large Group Name \par}
    
    \begin{table}[h!]
        \renewcommand{\arraystretch}{1.4}
        \centering
        \small
        \begin{tabular}{ r l }
            \textbf{Borga Haktan Bilen} & 22002733 \\
            \textbf{Vedat Eren Arıcan} & 22002643 \\
            \textbf{Berkan Şahin} & 22003211 \\
            \textbf{Berk Çakar} & 22003021 \\
            \textbf{Alp Ertan} & 22003912 \\
        \end{tabular}
    \end{table}
    
    % Document Type Header Table
    \begin{table}[h!]
        \renewcommand{\arraystretch}{1.5}
        \centering
        \begin{tabular}{ |>{\centering\arraybackslash}m{15.15cm}| }
            \hline
            \Large \textbf{Requirements Report} \\
            \small (Draft Version (pre-review)) \\
            \small \textbf{\today} \\
            \hline
        \end{tabular}
    \end{table}
    
    % Document begins here...
    
    \section{Introduction}
    
    LabConnect facilitates communication between students, TA's, tutors,
    and instructors. In the background, it is mainly a web application 
    (can be ported to Android possibly) that aims to assist CS introductory 
    courses in terms of organization and communication. Proposed ideas for 
    features include priority queuing for TA zoom rooms. For example, those 
    who have completed their labs can be tested using pre-defined (by TA or
    instructor) unit tests and ordered from most complete to least, in order
    to decrease waiting times for students who are done with their labs, and
    to optimize the process in general. TA's can also use the system to see
    previous versions of each student's code in a more practical way,
    similar to real version control managers in spirit. The style guidelines
    put forth by the instructors can be enforced automatically by parsing
    the student's sent code files. Much of the repetitive work that course
    staff need to do can be reduced substantially by automated actions,
    allowing TA's to allocate time for more hands-on help towards students.
    The student experience can be improved further by adding helpful
    features such as personal notes for students and so on.
    
    \section{Details}
    
    LabConnect is designed to contain three user interfaces for instructors, students,
    and assistants/graders. It will also contain a server side program where the submissions
    are stored and tested.
    
    \subsection{LabConnect - Instructor Side}
    
    \subsubsection{Prior to the lab}
    
    \begin{itemize}
        \item The instructor decides upon the name and the language of the assignment
        \item The instructor uploads the instructions either as a document, or as a Markdown
          or a plaintext file, in which case it will be rendered and displayed on the website
        \item The instructor writes the unit tests as input-output pairs and groups them if
          they wish. Some groups of unit tests can be hidden, in which case they won't be
          shown to the students prior to submission.
        \item The instructor can determine a time constraint for unit tests. If the execution
          of the code takes longer than the determined time, it will fail said test.
        \item The instructor determines a time frame for submissions. They can determine a
          seperate deadline for re-submissions if they wish.
        \item The instructor can assign students to assistants either at random or by hand.
          They can also choose to not assign assistants at all, in which case the students
          will be assigned to the assistants at the time of the lab, based upon the length
          of the queue.        
    \end{itemize}
    
    \subsubsection{During the Lab}
    
    \begin{itemize}
        \item The instructor will have better control over how code of the students' are tested.
        The instructor will be able to add more unit tests as the lab progresses. 
        The students' unit tests can be updated within the lab period so any mistakes made
        on the tests itself can be corrected this way.
    \end{itemize}
    
    \subsubsection{After the Lab}
    
    \begin{itemize}
        \item The common errors that are made by the student such as missing documentation 
        of the written code, conventions that aren't followed (naming conventions, styling 
        guidelines), will be detected by LabConnect. This data will be shared with the student
        and their instructor. The instructor can later on determine to act up on the most
        important mistakes that are made by the students.
        \item With the unit tests, instructors will be able to see which end cases of the program
        that the students mostly failed at. These unit tests can again show the common weaknesses
        of the programmers.
        \item The instructor will be able to assess their students properly by considering their
        performance in the lab. LabConnect will provide detailed performance of the student, 
        based on the mistakes they made, and overall unit test scores. This information can help
        the instructor to have more idea about their students since the data is properly organised
        and accessable.
    \end{itemize}
    
    \subsection{LabConnect - Student Side}
    
    \subsubsection{Prior to the Lab}
    
    % TODO
    
    \subsubsection{During the Lab}
    
    % TODO
    
    \subsubsection{After the Lab}
    
    % TODO
    
    \subsection{LabConnect - Grader/Assistant Side}
    
    \subsubsection{Prior to the Lab}
    
    % TODO
    
    \subsubsection{During the Lab}
    
    \begin{itemize}
        \item As stated, the program will be able to detect common errors made in student code. 
        Also, the unit tests can be arranged in a way they that they test specific skills that are
        required. During the lab, these properties of the program will increase the efficiency of the
        graders.
        \item The Grader can create pre-messages for the students if there is a highly repeated
        mistake. This way, they won't need to repeat the same small correction for each student,
        and this way they will find easier time to correct other unique mistakes that the students
        have made. The graders will find greater motivation on teaching the students proper techniques
        if they are freed from the repetitiveness of same mistakes. Of course the student can demand
        further help for their "repeated" mistake.
    \end{itemize}
    
    \subsubsection{After the Lab}
    % TODO
    
    \subsection{LabConnect - Server Side}

    \subsubsection{Prior to the Lab}
    
    % TODO
    
    \subsubsection{During the Lab}
    
    % TODO
    
    \subsubsection{After the Lab}
    
    % TODO
    
    \section{Summary \& Conclusions}
    
    % TODO
    
    \printbibliography
    
\end{document}
