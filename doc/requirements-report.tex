% Use either your LaTeX editor or latexmk to compile.

\documentclass[a4paper, 12pt]{article}

% Document quality things
\usepackage[utf8]{inputenc}
\usepackage{microtype, xcolor}
\usepackage{url, hyperref}
\hypersetup{colorlinks=true, linkcolor=blue, citecolor=black, urlcolor=blue}

% Setting margins
\usepackage[a4paper, left=2cm, right=2cm, top=1.75cm, bottom=1.75cm, includefoot]{geometry}

% Table helper packages
\usepackage{multirow, multicol}
\usepackage{makecell}
\usepackage{array}
%\usepackage{tabularx} % Not needed currently, but has a few nice options
%\usepackage{wrapfig} % Floating figures/tables

% Prevents spamming tedious newlines everywhere, also disables auto indentation, etc.
\usepackage[skip=0.75\baselineskip plus 2pt]{parskip}

% Self-explanatory
\usepackage{titlesec}
\titleformat{\section}[block]{\normalfont\scshape\Large}{\thesection}{1em}{}
\titleformat{\subsection}{\normalfont\large}{\thesubsection}{1em}{}

% Referencing
\usepackage[backend=bibtex, style=numeric-comp, sorting=none]{biblatex}
\addbibresource{bibliography.bib}

\begin{document}

    % Header Table
    \begin{table}[h!]
        \renewcommand{\arraystretch}{3}
        \centering
        \begin{tabular}{ | >{\raggedleft\arraybackslash}m{3cm} l >{\raggedleft\arraybackslash}m{3cm} m{3cm} | }
            \hline
            \Huge CS 102 & \textit{Spring 2020/21} & \multirow{2}{*}{\makecell{Project\\Group}} & \multirow{2}{*}{\textbf{\Huge G2C}} \\
            \makecell[r]{Instructor:\\Assistant:} & \makecell[l]{\textbf{Aynur Dayanık}\\\textbf{Haya Shamim Khan Khattak}} & & \\
            \hline
        \end{tabular}
    \end{table}

    % Grading Table
    \begin{table}[h!]
            \renewcommand{\arraystretch}{1.4}
            \centering
            \footnotesize
            \begin{tabular}{ l p{1.5cm} | p{1.5cm} | }
                \hline
                \multicolumn{1}{|c|}{\textbf{Criteria}} & \multicolumn{1}{c|}{\textbf{TA/Grader}} & \multicolumn{1}{c|}{\textbf{Instructor}} \\ \hline
                \multicolumn{1}{|p{10.5cm}|}{Presentation} &  &  \\[10ex] \hline
                \multicolumn{1}{r|}{\textbf{Overall}} &  &  \\
                \cline{2-3}
            \end{tabular}
    \end{table}

    % Project Information Header
    {\centering\Huge \bfseries \raisebox{0.5ex}{\texttildelow} LabConnect \raisebox{0.5ex}{\texttildelow} \par}

    %{\centering\large Group Name \par}

    \begin{table}[h!]
        \renewcommand{\arraystretch}{1.4}
        \centering
        \small
        \begin{tabular}{ r l }
            \textbf{Borga Haktan Bilen} & 22002733 \\
            \textbf{Vedat Eren Arıcan} & 22002643 \\
            \textbf{Berkan Şahin} & 22003211 \\
            \textbf{Berk Çakar} & 22003021 \\
            \textbf{Alp Ertan} & 22003912 \\
        \end{tabular}
    \end{table}

    % Document Type Header Table
    \begin{table}[h!]
        \renewcommand{\arraystretch}{1.5}
        \centering
        \begin{tabular}{ |>{\centering\arraybackslash}m{15.15cm}| }
            \hline
            \Large \textbf{Requirements Report} \\
            \small (version 2.1) \\
            \small \textbf{\today} \\
            \hline
        \end{tabular}
    \end{table}

    % Document begins here...

    \section{Introduction}

    LabConnect facilitates communication between students, TA's, tutors,
    and instructors. In the background, it is mainly a web application
    (can be ported to Android possibly) that aims to assist CS introductory
    courses in terms of organization and communication. Proposed ideas for
    features include priority queuing for TA zoom rooms. For example, those
    who have completed their labs can be tested using pre-defined (by TA or
    instructor) unit tests, if students pass the tests successfully then they will be
    ordered by the number of visits to TA in the same session, in order to decrease
    waiting times for the students who are waiting from the beginning, and
    to optimize the process in general. TA's can also use the system to see
    previous versions of each student's code in a more practical way,
    similar to real version control managers in spirit. The style guidelines
    put forth by the instructors can be enforced automatically by parsing
    the student's sent code files. Much of the repetitive work that course
    staff need to do can be reduced substantially by automated actions,
    allowing TA's to allocate time for more hands-on help towards students.
    The student experience can be improved further by adding helpful
    features such as personal notes for students and so on.

    \subsection{Related Work}

    As a part of their curriculum most of the big engineering instutions and universities have hands-on laboratory sessions, that are mandatory
    for students to attend them. Each lab session, every student has to submit approximately 9 to 12 assignments. Additionally, every student takes lab tests (up to three lab tests).
    Cumulatively the amounts of submissions are approximately 10000 per semester. Even if there were 20 evaluators, each evaluator would need to take care of almost 500 assignments.
    Without the help of automation, evaluators spend most of their time grading and testing work rather than creating more useful assignments for students \cite{Mandal2007}.

    Some of the positive sides of the automated grading are syntactic correctness, maintainability and efficiency.
    Moreover, automated grading systems weighs down the lack of objectivity in the conventional grading and these systems
    can track all of the building process. In the contemporary context, these kinds of automated grading systems are commonly used by
    some commercial competitive programming and recruitment sites (for instance, HackerRank, TopCoder and HackerEarth) \cite{RestrepoCalle2018}.

    At the same time, the quantity of the feedback given by teachers is directly proportional to the effectiveness of the student's work that is done
    after the class. The students have an easier time studying weak areas if they are informed about their
    performance properly. The autonomous natured variant of instructor feedback frees everyone from manual assessment,
    which is a time consuming and unproductive task \cite{RestrepoCalle2018-2}.
    
    If there are many more students than there are teachers, this can hinder the process of manual assessment of the student's code.
    Although this problem can be solved by increasing the amount of teachers, this solution can harm the university economically.
    The idea of automated assesment comes into play with its ability to solve this problem without increased costs.
    There is much evidence that shows that student code is evaluated more properly using automatic assesment rather than using pen\&paper testing.
    Evidence also shows that automated assesment reduces the rate of dropout in programming majors \cite{Gordillo2019}.

    The effectiveness of different testing procedures are as follows:
    Unit testing is able to catch 50\% of the errors.
    Integration testing is able to catch 40\% of the errors.
    Regression testing is able to catch 30\% of the errrors \cite{Fenton2018}.

    The evaluation system decreases the work load of the evaluation process since it makes the evalution process easier.
    Just as the system has automatic testing, it also comes with automatic grading. Additionally, it manages students by aiding
    grouping and scheduling issues, helping with the students checking in to lab sessions, the deadlines and overall project grades.
    As the system is ideally available all the time, the students will be able to get feedback instantly on the correctness of
    their solutions. This way, students keep working on their work at their own pace \cite{Nogueira2011}.

    And finally, it was seen that the approach used for automation of assignments in education is used in DevOps fields in a similar way \cite{Faber2020}.

    \section{Details}

    LabConnect is designed to contain three user interfaces for instructors, students,
    and assistants/graders. It will also contain a server side program where the submissions
    are stored and tested.

    \subsection{LabConnect - Instructor Side}

    \subsubsection{Task Definition Stage}

    \begin{itemize}
        \item The instructor decides upon the name and the language of the assignment.
        \item The instructor uploads the instructions either as a document, or as a Markdown
          or a plaintext file, in which case it will be rendered and displayed on the website.
        \item The instructor writes the unit tests as input-output pairs and groups them if
          they wish. Some groups of unit tests can be hidden, in which case they won't be
          shown to the students prior to submission.
        \item The instructor provides a tester class that tests the code provided by the student
        by calling the methods. This class is the one that will be run by the server. This allows students
        to be more flexible in their own main class.
        \item The instructor can determine a time constraint for unit tests. If the execution
          of the code takes longer than the determined time, it will fail said test.
        \item The instructor determines a time frame for submissions. They can determine a
          seperate deadline for re-submissions if they wish.
        \item The instructor can assign students to assistants either at random or by hand.
          They can also choose to not assign assistants at all, in which case the students
          will be assigned to the assistants at the time of the lab, based upon the length
          of the queue.
        \item The instructor can either define PAIN-style (Proficient, Acceptable, Incomplete, Nothing)
          tiers for grading or can define certain criteria for assistants to enter a numeric value. If they
          decide to use tiers, they can define certain tiers (such as Proficient and Acceptable) as ``complete'',
          in which case the submission of a student receiving said grades will automatically be classified as complete.
          If they decide to use numeric grades, they can define a certain threshold that needs to be exceeeded for a
          submission to be classified as complete.
    \end{itemize}

    \subsubsection{Task Update Stage}

    \begin{itemize}
        \item The instructor will have better control over how students' codes are tested.
        The instructor will be able to add more unit tests as the lab progresses.
        The students' unit tests can be updated within the lab period so any mistakes made
        on the tests themselves can be corrected this way.
    \end{itemize}

    \subsubsection{After the Lab}

    \begin{itemize}
        \item The common errors that are made by students such as missing documentation (see JavaDoc)
        of the written code, and predefined conventions that aren't followed (naming conventions, styling
        guidelines, etc.), will be detected by LabConnect. This data will be shared with the student
        and their instructor. The instructor can later on determine to act up on the most
        important mistakes that are made by the students.
        \item With the unit tests, instructors will be able to see which test cases of the program
        the students mostly failed at. These unit tests can again reveal the common weaknesses
        of programmers.
        \item The instructor will be able to assess their students properly by considering their
        performance in the lab. LabConnect will provide detailed performance of the student,
        based on the mistakes they made, and overall unit test scores. This information can help
        the instructor achieve a better grasp of their students since the data is properly organised
        and accessible.
    \end{itemize}

    \subsection{LabConnect - Student Side}

    \subsubsection{Code Submission Process}

    \begin{itemize}
      \item The students will upload, one by one, the individual files specified by the instructor during the task specification
      stage. That way, they won't have to reupload the whole assignment if only one specific file is problematic.
      \item Once the student submits their assignment, the automated tests and checks are run on the server side.
      \item If the submission fails any test, it is marked as incomplete and the student is redirected to the revision stage.
      \item Otherwise the student is placed into the code review queue.
    \end{itemize}

    \subsubsection{Code Review Queue}

    \begin{itemize}

      \item When the student's submission passes all (or possibly most) unit tests, they are placed into the code review queue.
      \item Once the student is queued, they can see how many students are waiting in front of them. Their place in the queue is
      stored on the server side, so in the event of a momentary disconnection, they won't lose their place.
      \item Once the student reaches the very top of the queue, they have 30 seconds to click an ``I am ready''
      button to receive the meeting link for their TA. Failure to do so will result in the assumption that the student is not ready, and they
      are placed at the end of the current queue.
    \end{itemize}

    \subsubsection{Code Revision Process}


    \begin{itemize}
      \item If the student's submission is marked as incomplete either by their TA or automatically by the system, they are directed
      to the revision stage.
      \item In this stage, the student either receives written feedback from their TA or they receive a message generated server-side explaining
      what went wrong during the tests. In case of a runtime exception/compiler error, the stack trace is also provided and the lines causing the
      problem are highlighted. Code can also be highlighted manually by the TA, as described in the Code Review Process section of the TA interface.
      \item Once the student is confident that they have fixed the problem, they can re-submit their code as described in the Code Submission Process
      section of the Student interface.
    \end{itemize}

    \subsection{LabConnect - Grader/Assistant Side}

    \subsubsection{Code Review Process}

    \begin{itemize}
        \item At the start of the lab, the assistants will enter their meeting links to the system.
        This will allow for the links to be distributed to the students when it is their turn.
        \item After the assistant enters their meeting link, they will be directed to the code review interface
        This interface will contain information about the student, as well as the source code submitted by the student
        and a field to write feedback about the submission.
        \item The assistant/grader can choose to browse the code submitted by the student from their computer without wasting bandwidth
        and time with screen sharing. They can also write feedback referencing specific lines in the code to make their
        feedback clearer.
        \item After evaluating the code, the assistants can either give the assignment a score or pick
          from the tiers determined by the instructor. If the determined grade or tier is within the satisfactory
          threshold, the student's assignment is marked as complete. Otherwise, the assignment is returned to the
          student with the feedback written by the TA.
        \item Once the assistant finishes evaluating a submission, they need to manually confirm that they are
          ready for the link to be revealed to the next student in queue. This allows the assistants to take short
          breaks if necessary.
    \end{itemize}

    \subsubsection{Task Revision Process}

    \begin{itemize}
      \item As stated, the program will be able to detect common errors made in the student's code.
      Also, the unit tests can be arranged in a way such that they test specific skills that are
      required. During the lab, these properties of the program will increase the efficiency of the
      graders.
      \item The Grader can create announcements for the students if there is a highly repeated
      mistake. This way, they won't need to repeat the same small correction for each student,
      and they will find it easier to correct other unique mistakes that the students
      have made. The graders will find greater motivation on teaching the students proper techniques
      if they are freed from the repetitiveness of correcting the same mistakes. Of course, the student can demand
      further help for their "repeated" mistakes.
    \end{itemize}

    \subsection{LabConnect - Server Side}

    \subsubsection{Task Definition Stage}
    \begin{itemize}
      \item The distribution of students to the TAs, that should be uploaded by instructor (probably once at the
      beginning of the semester), will be stored in the database.
      \item For each TA, a unique identification number will be generated and the students who are allocated to a
      TA will get the identification number of that TA.
      \item For each lab, every student will have their own git repository (created by the system) for the storage of the
      code.
      \item The test cases (inputs \& outputs), that are uploaded by instructor, will be stored in the database.
    \end{itemize}

    \subsubsection{Post-Submission Stage}

    \begin{itemize}
      \item Each submission will be stored in a version control system. This allows LabConnect to provide easy access to
      submission history for TAs and instructors.
      \item If the current submission doesn't differ from the last submission, it will be automatically rejected.
      \item When a student makes a submission, the server-side program will compile the source code in the submission
      files if the language picked by the instructor is a compiled language.
      \item Then the server-side program runs either the resulting binary if the language is a compiled language, or the
      interpreter for the language if it is an interpreted one.
      \item If there's an error during the compilation or interpretation/execution stages, the submission is automatically failed and
      the error message is shown to the student.
      \item Using I/O redirection, the input part of the unit tests are fed to the standard input of the tester program and the output is captured.
      This allows for easy multi-language support and doesn't require students to deal with file I/O while writing the program.
      \item If the output of the program doesn't match the expected results, the submission is automatically failed and the input, expected
      output and actual output for the specific unit test is shown to the student.
      \item The submission is run separately for each unit test. These runs are timed and the results for each unit test is stored in a database.
      If the instructor specified a time limit for tests, the test in question will automatically fail once the time limit is reached.
    \end{itemize}

    \subsubsection{Queue Management}
    \begin{itemize}
      \item Once a submission passes all the automated tests, the student who made the submission will be added to the queue for their TA.
      \item The students that didn't receive any manual code review are prioritized over those that received one. This allows each student to
      get their code reviewed at least once during the labs.
      \item If the TA's were not assigned to students by the instructor, the server will pick a TA on the spot based on the queue length. This
      behavior will help make the workloads of the TA's more balanced.
    \end{itemize}

    \subsubsection{Statistics Generation}

    \begin{itemize}
      \item After the lab is finished, the data acquired from the unit tests, the error detection algorithm, etc. will be used
      to create statistics about the students' performace. The server will be used to maintain this data for further use.
    \end{itemize}

    \section{Summary \& Conclusions}

    The main purpose of LabConnect is to provide students and instructors a place where everyone will have an easier time
    collaborating with each other properly. Especially in online education, instructors can have a hard time monitoring students'
    performance on labs. With LabConnect, instructors can shape, and take control of, their classes according to the response given to the lab sessions.

    \pagebreak

    \printbibliography
\end{document}
