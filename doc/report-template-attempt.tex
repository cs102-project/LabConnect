% The order to compile is:
% pdflatex -> bibtex -> pdflatex
% Doing this process with your current working directory
% pointing to where all the tex/bib/etc. files are may help.

\documentclass[a4paper, 12pt]{article}

% Document quality things
\usepackage[utf8]{inputenc}
\usepackage{microtype, xcolor}
\usepackage{url, hyperref}
\hypersetup{colorlinks=true, linkcolor=blue, citecolor=red, urlcolor=blue}

% Setting margins
\usepackage[a4paper, left=2cm, right=2cm, top=1.75cm, bottom=1.75cm, includefoot]{geometry}

% Table helper packages
\usepackage{multirow, multicol}
\usepackage{makecell}
\usepackage{array}
%\usepackage{tabularx} % Not needed currently, but has a few nice options
%\usepackage{wrapfig} % Floating figures/tables

% Prevents spamming tedious newlines everywhere, also disables auto indentation, etc.
\usepackage[skip=0.75\baselineskip plus 2pt]{parskip}

% Self-explanatory
\usepackage{titlesec}
\titleformat{\section}[block]{\normalfont\scshape\Large}{\thesection}{1em}{}
\titleformat{\subsection}{\normalfont\large}{\thesubsection}{1em}{}

% Referencing
\usepackage[backend=bibtex, style=numeric-comp, sorting=none]{biblatex}
\addbibresource{bibliography.bib}

\begin{document}
    
    % Header Table
    \begin{table}[h!]
        \renewcommand{\arraystretch}{3}
        \centering
        \begin{tabular}{ | >{\raggedleft\arraybackslash}m{3cm} l >{\raggedleft\arraybackslash}m{3cm} m{3cm} | }
            \hline
            \Huge CS 102 & \textit{Spring 2020/21} & \multirow{2}{*}{\makecell{Project\\Group}} & \multirow{2}{*}{\textbf{\Huge G2C}} \\
            Assistant: & \textbf{Haya Shamim Khan Khattak} & & \\
            \hline
        \end{tabular}
    \end{table}
    
    % Grading Table
    \begin{table}[h!]
            \renewcommand{\arraystretch}{1.4}
            \centering
            \footnotesize
            \begin{tabular}{ l p{1.5cm} | p{1.5cm} | }
                \hline
                \multicolumn{1}{|c|}{\textbf{Criteria}} & \multicolumn{1}{c|}{\textbf{TA/Grader}} & \multicolumn{1}{c|}{\textbf{Instructor}} \\ \hline
                \multicolumn{1}{|p{10.5cm}|}{Presentation} &  & Aynur Dayanık \\[10ex] \hline
                \multicolumn{1}{r|}{\textbf{Overall}} &  &  \\
                \cline{2-3}
            \end{tabular}
    \end{table}
    
    % Project Information Header
    {\centering\Huge \bfseries \raisebox{0.5ex}{\texttildelow} LabConnect \raisebox{0.5ex}{\texttildelow} \par}
    
    {\centering\large Group Name \par}
    
    \begin{table}[h!]
        \renewcommand{\arraystretch}{1.4}
        \centering
        \small
        \begin{tabular}{ r l }
            \textbf{Borga Haktan Bilen} & 22002733 \\
            \textbf{Vedat Eren Arıcan} & 22002643 \\
            \textbf{Berkan Şahin} & 22003211 \\
            \textbf{Berk Çakar} & 22003021 \\
            \textbf{Alp Ertan} & 22003912 \\
        \end{tabular}
    \end{table}
    
    % Document Type Header Table
    \begin{table}[h!]
        \renewcommand{\arraystretch}{1.5}
        \centering
        \begin{tabular}{ |>{\centering\arraybackslash}m{15.15cm}| }
            \hline
            \Large \textbf{Report Type} \\
            \small (report subtype/version) \\
            \small \textbf{\today} \\
            \hline
        \end{tabular}
    \end{table}
    
    % Document begins here...
    
    \section{Introduction}
    
    View this template in ``Print Layout" form. To use it, begin by editing 
    the preceding section to include information related to your project \& 
    the report you are writing (for help, press F1 when the text cursor is 
    in a field.) 
    
    Using Styles
    
    As far as possible, do not change any of the formatting, but rather use the 
    existing styles. For example, place the cursor in the text ``Using Styles" 
    above. Notice the Style is ``Paragraph". Try changing it to ``Heading 2", then to 
    ``Heading 1". Note the numbering of subsequent sections is changed automatically.
    
    \section{Details}
    
    The real work goes here! Replace section titles with something relevant to your report.
    
    \subsection{Subsections}
    
    \subsection{Using References}
    
    Don’t forget to acknowledge any sources that you make use of. Claiming other 
    people's work \& ideas as your own is considered cheating and carries severe 
    penalties. Make it very clear what is your work and what help, words, ideas, etc., 
    you have taken from elsewhere. Be sure to put quotation marks around any sections 
    of text you copy from elsewhere and add a reference to the original source 
    (see \cite{plagiarism} for general information and \cite{note1} and \cite{note2} for examples of in-formation 
    required for book, journal and web-based sources). 
    
    You can insert references in the text by selecting “Insert|Reference-Footnotes... Endnotes.” 
    This template uses sequential numbers for references, the most common format used for 
    technical articles. The template includes a macro, ``Create\_Reference" which should insert 
    a link in-to the text and a corresponding entry into the References section at the end of 
    the document, which you can then edit. You can invoke the macro by pressing Control-R 
    (but you may need to ``Tools|Unprotect Document" and/or Enable-Macros first!) 
    Note: newer versions of Word may now include this reference style-check the help.
    
    \subsection{And Outlines?}
    
    Once you understand the basics, you may want to switch to ``Outline" view to sort out your ideas 
    before returning to the ``Page Layout" view to write the actual content. If you learn to use styles, 
    outlines and endnotes properly, then Word sorts out the numbering, formatting, etc. for you. 
    Try inserting and deleting some of the references from the text and notice again how the other 
    numbers change automatically. Having the machine do the layout and such numbering automatically, 
    enables you to concentrate on what is really important, the content. 
    Neat and very professional looking, eh?
    
    
    \section{Summary \& Conclusions}
    
    And finally... Don’t forget that Word can help to check your spelling (and grammar!)
    
    Maintaining lists of research references that can be reused when writing journal articles 
    can be a real pain, especially when citation styles vary so much from journal to journal. 
    When you have time, I suggest you look at reference managers 
    (e.g., JabRef for BibTeX, or websites such as CiteSeer), as well as other document creation 
    options (e.g., LyX, \LaTeX and OpenOffice.)
    
    Good Luck.
    
    
    
    \printbibliography
    
    
        
\end{document}
